\documentclass[11pt,letterpaper]{minimal/moderncv}
\moderncvcolor{black}
\moderncvstyle[nodetails]{casual}
\usepackage[margin=1in]{geometry}
\usepackage{bibentry}
\usepackage{natbib}
\newcommand{\enquote}[1]{{``#1''}}
\newcommand{\cvbibentry}[2]{\cvitem{#2}{\leavevmode\nobreak\bibentry{#1}.\vspace*{6pt}}}
\usepackage[dvipsnames]{xcolor}
\usepackage{hyperref}
\hypersetup{colorlinks=true,linkcolor=Blue,urlcolor=Blue}
\newcommand{\xx}{\textcolor{white}{2022}}
\newcommand{\xxx}{\textcolor{white}{--2022}}
\newcommand{\jog}{\hspace*{2ex}}
\renewcommand{\cventry}[2]{\cvitem{#1}{\begin{minipage}[t]{\maincolumnwidth}#2\end{minipage}}}
\newcommand{\bolditem}[1]{\cvitem{}{\textbf{#1}}}
\renewcommand{\th}{$^\text{th}$}
\newcommand{\st}{$^\text{st}$}

\usepackage[utf8]{inputenc}
\usepackage[T1]{fontenc}
\usepackage{fdsymbol}
\usepackage{lmodern}



% document language
\usepackage[english]{babel}

\name{Tamara L.}{Mitchell}

\usepackage[inline]{enumitem}
\setlist{nosep,labelindent=0ex,itemindent=0ex}

% lengths
\setlength{\parindent}{0pt}
\setlength{\parskip}{.5ex}
\setlength{\hintscolumnwidth}{0.125\textwidth}
\setlength{\separatorcolumnwidth}{0.025\textwidth}
\setlength{\maincolumnwidth}{.85\textwidth}

\renewcommand{\section}[1]{  \par\addvspace{15pt}%
  \parbox[t]{\hintscolumnwidth}{\strut\raggedleft\raisebox{4pt}%
  {\rule{\hintscolumnwidth}{2pt}}}%
  \hspace{\separatorcolumnwidth}%
  \textsc{\Large #1}\par\nobreak\addvspace{6pt}%
}

\renewcommand*{\cvitem}[2]{%
  \begin{tabular}{@{}p{\hintscolumnwidth}@{\hspace{\separatorcolumnwidth}}%
    p{\maincolumnwidth}@{}}%
    \raggedleft {#1} &  {#2}%
    \vspace*{3pt}
  \end{tabular}%
}
\newcommand*{\cvitemhang}[2]{%
  \begin{tabular}{@{}p{\hintscolumnwidth}@{\hspace{\separatorcolumnwidth}}%
    p{\maincolumnwidth}@{}}%
    \raggedleft {#1} &  {#2}%
  \end{tabular}%
}



\begin{document}
\maketitle
\vspace*{-20pt}
\fancyfoot[c]{{\itshape Tamara L.\ Mitchell --- Version: \today}}


\section{Contact Information}
\cvitem{}{%}
\begin{tabular}{@{}p{3in}p{2.75in}p{.25in}@{}}
Department of French, Hispanic  & \emph{email:}
\href{mailto:tamara.mitchell@ubc.ca}{tamara.mitchell@ubc.ca}\\
\; and Italian Studies & \emph{www:}
\href{https://tamaraleemitchell.github.io/}{tamaraleemitchell.github.io}\\
University of British Columbia 
\\
824 Buchanan Tower, 1873 E Mall\\ Vancouver, BC V6T 1Z1
\end{tabular}
}


\section{Academic employment}

\cventry{2019--\xx}{\textbf{Assistant Professor of Spanish}, Department of French, Hispanic and
Italian Studies, University of British Columbia, Vancouver, Canada}

\section{Education}

\cventry{2019\xxx}{Doctor of Philosophy in Hispanic Literatures\\ 
\textbf{Indiana University}, Bloomington IN, USA\\
{\small \jog\emph{Dissertation:} ``Neoliberalism, Post-Nationalism, and the Ghosts of
Lefts Past: Reading Roberto Bolaño and Horacio Castellanos Moya on
Politics and the Literary Tradition''\\ \jog\emph{Advisor:} Patrick Dove}}

\vspace{3pt}
\cventry{2009\xxx}{Master of Arts in Spanish Literature\\ \textbf{Kansas
University}, Lawrence KS, USA}

\vspace{3pt}
\cventry{2006\xxx}{Bachelor of Arts in English Literature; Spanish\\
\textbf{Washburn University}, Topeka KS, USA\\
\jog{\small Summa cum laude}}

\section{Research and teaching specializations}

\cvitem{}{20\th \& 21\st-Century Latin American Studies $|$ Critical
Theory and Philosophy $|$
Sound Studies $|$ Mexican Literature \& Culture $|$
Neoliberalism/Globalization $|$ Comparative Literature $|$ 
Central American Literature \& Culture $|$ Border and Diaspora Studies $|$ Politics \& Nationalism}

\section{Peer-reviewed publications}

\cvitem{}{\textbf{Monograph}}

\vspace{6pt}
\cvitem{}{\emph{Novel Distortions: Postnational Form in Mexican and Central
American Narrative}. (100\%, under contract, University of Pittsburgh
Press, Illuminations Series)}

\vspace{6pt}
\cvitem{}{\textbf{Special Issues for a Journal}}

\cvitem{2023}{\emph{Latin American Literary Aurality}, co-edited with Prof. Amanda M.
Smith (UC Santa Cruz). \emph{Revista de Estudios Hispánicos}, vol. 53,
no. 3. Oct. 2023.}

\cvitem{2025} {Editor, Review Dossier. ``Sound, Listening and Music in Latin American
Criticism.'' \emph{Journal of Latin American Cultural Studies} (invited
August 2024; proposal accepted February 2025).}

\cvitem{2026}{\emph{Los espacios sonoros: El silencio como expresión en la literatura
y cultura latinoamericana contemporánea}, co-edited with Isabella
Vergara Calderón (Princeton U). \emph{Arizona Journal of Hispanic
Cultural Studies}, vol. 30, 2026 (proposal accepted December 2024).}

\vspace{6pt}
\cvitem{}{\textbf{Articles}}

\cvitem{2025+}{``Señorita Bovary in San Salvador: Speculative Realism and the Novel in
El Salvador.'' \emph{Comparative Literature} (accepted August 2025).}

\cvitem{}{``Black Dawn: Failed Allegory and Opacity against Gendered
Violence in Roberto Bolaño's `La parte de Fate.'\,'' \emph{PMLA} (submitted June
2025, revise and resubmit)}

\cvitem{2023}{``Sounding Out the Text: Approaches to Latin American Literary
Aurality.'' with Prof. Amanda M. Smith (UC Santa Cruz). \emph{Revista de
Estudios Hispánicos}, vol. 53, no. 3, October 2023, pp. 365-378.}

\cvitem{2021}{``From Ratiocination to Globalization: Poe, Borges, Bolaño and the
\emph{Complot} of the \emph{Novela Negra Mexicana}.'' \emph{CR: The New
Centennial Review}, vol. 21, no. 3, December 2021, pp. 105-33.}

\cvitem{}{``A Narrative Vaivén: \emph{Lucha libre} and the Modern Nation
Unready-to-hand in Horacio Castellanos Moya\textquotesingle s \emph{La
sirvienta y el luchador}.'' \emph{Modern Language Notes}, vol. 136, no.
2, March 2021, pp. 270-91.}

\cvitem{}{``Broken Bodies, Broken Nations: Roberto Bolaño on Neoliberal Logic and
(Un)Mediated Violence.'' \emph{Revista de Estudios Hispánicos}, vol. 55,
no. 1, March 2021, pp. 189-211.}

\cvitem{2019}{``Escatología y marginalización en la literatura andina: Las
porosas fronteras sociopolíticas en \emph{Los ríos profundos} de José María
Arguedas.'' \emph{Revista Canadiense de Estudios Hispánicos}, vol. 43, no. 2,
December 2019, 425-47.}

\cvitem{}{``Geopoetics, Geopolitics, and Violence: (Un)Mapping Daniel Alarcón's
\emph{Lost City Radio}.'' \emph{Latin American Perspectives}, vol. 46,
no. 5, September 2019, pp. 186-201.}

\cvitem{2018}{``Carving Place out of Non-Place: Luis Rafael Sánchez's `La guagua
aérea' and Post-National Space.'' \emph{Chasqui: revista de literatura
latinoamericana}, vol. 47, no. 1, May 2018, pp. 275-92.}


\vspace{6pt}
\cvitem{}{\textbf{Book Chapters}}

\cvitem{2025+}{``From Plaintive to Plaintiff: Accessing Rights in Neoliberal
Globalization,'' \emph{Weeping Women: La Llorona in Modern Latina and
Chicana Lore}, eds. Norma Elia Cantú and Kathleen Alcalá, Trinity
University Press (forthcoming).}

\cvitem{}{``Forced Displacement and Aurality in Mexico: Intertextual Echoes and
More-than-Human Sound in Suzette Celaya Aguilar's \emph{Nosotras},''
\emph{Contemporary Migrations in Latin American Literature}, edited by
Cecily Raynor and Lara Bourdin. Co-authored with Tesi Aguirre Alfaro.
(submitted August 2025, under review).}

\cvitem{2023}{``Rewriting the Militant Left: Untranslatability and Dissensus in
Horacio Castellanos Moya,'' \emph{Central American Literature as World
Literature}, ed. Sophie Esch, Bloomsbury Academic \emph{World Literature
Series}, October 2023, pp. 155-71.}

\cvitem{2022}{``The Regional Novel and the Novel of the Mexican Revolution on
Common Ground.'' \emph{Oxford Handbook of the Latin American Novel}. Co-authored
with Amanda M. Smith, UC Santa Cruz, eds. Ignacio López-Calvo and Juan E. De
Castro. November 2022, pp. 75-92.}

\cvitem{}{``Migration and Diaspora: Central American Literature Beyond the
Isthmus,'' \emph{Teaching Central American Literature in a Global
Context}, edited by Mónica Albizúrez and Gloria E. Chacón, Modern
Language Association of America, 2022, pp. 262-74.}



\section{Review essays}

\cvitem{2025}{``Aural Identities: Chicanx and Mexican Sonic Media and Performance.'' \emph{Journal of Latin American Cultural
Studies} (December 2025).}


\cvitem{}{``Ephemeral Aurality: Sonic Media and Performance in Mexico.''
\emph{Journal of Latin American Cultural Studies} (December 2025). Books reviewed:}
\cventry{}{\begin{tabular}{p{.05\textwidth}p{.8\textwidth}}
&\emph{Echoes from the Borderlands} by Valeria Luiselli, Ricardo
Giraldo, Leo Heiblum (Dia Art Foundation, 2025)\\
&\emph{Sonic Strategies: Performing Mexico's War on Drugs, Mourning, and
Feminicide} by Christina Baker~(Vanderbilt UP, 2024)\\
&\emph{The Archive and the Aural City: Sound, Knowledge, and the Politics
of Listening} by Alejandro Madrid (Duke UP, 2025)\\
&\emph{Mexican Waves: Radio Broadcasting along Mexico's Northern Border,
1930--1950} by Sonia Robles (U of Arizona P, 2019)\end{tabular}}

\cvitem{2022}{``Listening in/to Literature.'' Commissioned Thematic Review,
\emph{Latin American Research Review}, vol. 85, no. 1, October 2022, pp.
215-25. Books reviewed:}
\cventry{}{\begin{tabular}{p{.05\textwidth}p{.8\textwidth}}
&\emph{Tropical Riffs: Latin America and the Politics of Jazz} by Jason
Borge (Duke UP, 2018)\\
&\emph{The Cry of the Senses: Listening to Latinx and Caribbean Poetics}
by Ren Ellis Neyra (Duke UP, 2020)\\
&\emph{Hearing Voices: Aurality and New Spanish Sound Culture in Sor
Juana Inés de la Cruz} by Sarah Finley (U of Nebraska P, 2019)\\
&\emph{Writing by Ear: Clarice Lispector and the Aural Novel} by Marília
Librandi (U of Toronto P, 2018)\\
&\emph{The Senses of Democracy: Perception, Politics, and Culture in
Latin America} by Francine R. Masiello (U of Texas P, 2018)\\
&\emph{Sonar. Navegación / localización del sonido en las prácticas
artísticas del siglo XX} by Luz María Sánchez Cardona (UAM Juan Pablos
Editor, 2018).\end{tabular}}

\section{Book Reviews}

\cvitem{2021}{\emph{The Vanishing Frame: Latin American Culture and Theory in the
Postdictatorial Era} (U of Texas P, 2018) by Eugenio Claudio Di Stefano,
\emph{Studies in Twentieth and Twenty-First Century Literature}, 45.1
(August 2021): Article 33.}

\cvitem{}{\emph{Modernity at Gunpoint: Firearms, Politics, and Culture in Mexico
and Central America} (U of Pittsburgh P, 2018) by Sophie Esch.
\emph{Revista de Literatura Mexicana Contemporánea}, vol. 27, no. 82
(Spring 2021): pp. 133-36.}

\cvitem{2018}{\emph{Chicana/o Remix: Art and Errata Since the Sixties} (NYU P, 2017)
by Karen Mary Davalos. \emph{Chiricú Journal: Latina/o Literatures,
Arts, and Cultures} 2.2 (May 2018): 229-31.}

\cvitem{2017}{\emph{Pliegues del yo: Cuatro estudios sobre escritura autobiográfica en
Hispanoamérica} (Cuarto Propio, 2015) by Sergio R. Franco. \emph{Revista
de Estudios Hispánicos} 51.3 (October 2017): 718-21.}

\section{Invited, Non-Refereed}

\cvitem{2016}{``Natalia Almada: The Sound \& the Image,'' \emph{Chiricú Journal:
Latina/o Literatures, Arts, and Cultures}, vol. 1, no. 1, September
2016, pp. 129-34.}

\cvitem{2013}{Translations of poetry by Conceição Evaristo (Brazilian
Portuguese), \emph{Revista Hiedra}, vol. 1, no. 1, Fall 2013, pp. 66-68.}

\section{Manuscript in progress}

\cvitem{}{\emph{Sounds of the Capitalocene: Extraction and Aurality in Mexican
Literature} (monograph, 50\%)}

\section{Digital and public humanities research projects}

\cvitem{2025}{``\href{https://soundstudiesblog.com/2025/03/03/clapping-back-responses-from-sound-studies-to-censorship-silencing/}{Clapping
Back}: Responses from Sound Studies to Censorship \& Silencing.''
\emph{SoundingOut!}, collective post as part of Modern Language
Association MS Sound Forum Executive Committee (March)}

\cvitem{2022--24}{PI and Director, Sound and the Humanities Research
\href{https://sound.arts.ubc.ca/}{Cluster}}

\cvitem{2020}{PI and Convenor, Latin American Sound Studies Collective
\href{https://docs.google.com/document/d/1rd5NcA06JHf0Bk0vVV17LW7F95MfVgoZCUhUkWMelmE/edit?usp=sharing}{Bibliography},
Working Group}

\cvitem{2020}{Steering Committee, Discussion Facilitator,
\href{https://blogs.ubc.ca/virtualkoerners/}{Virtual Koerner's} Working
Group}

\cvitem{2020}
{``\href{https://medium.com/the-humanities-in-transition/in-defense-of-wandering-podcasting-as-a-pedagogical-tool-f3daf52746aa}{In
Defense of Wandering}: Podcasting as a Pedagogical Tool.'' \emph{The
Humanities in Transition} (June)}

\cvitem{2020}{Open-access podcast
\href{https://blogs.ubc.ca/course0935bb2c1345dfa4e91d0701421d97f5c03a0045/category/lecture/}{episodes}
on Latin American Indigenous Foodways}

\cvitem{2020}{``\href{https://blogs.ubc.ca/indigenouslabour/}{Podcasting} as a
Learning Outcome,'' Open-access pedagogy
\href{https://blogs.ubc.ca/indigenouslabour/tutorials/}{materials} \&
Student episodes}

\cvitem{2008--10}{Editor, \href{http://acceso.ku.edu/}{\emph{ACCESO}} digital
textbook: Hispanic languages and cultures, University of Kansas}

\section{Select grants, fellowships and awards}

\cvitem{2024--29}{PI, SSHRC Insight Grant, ``Sounds of the Capitalocene:
Extraction and Aurality in Mexican Literature,'' \$76,959}

\cvitem{2025--27}{Co-Investigator (with PI Ben Bryce, UBC History), VPRI
Catalyzing Research Cluster, ``Latin American and Caribbean
Landscapes,'' \$200,000}

\cvitem{2024--25}{Translation \& Dissemination Grant, Latin American Landscapes,
UBC, \$2600}

\cvitem{2023--25}{Co-Investigator (with PI Ben Bryce, UBC History), VPRI
Catalyzing Research Cluster, ``Latin American Landscapes,'' \$100,000}

\cvitem{2022--24}{PI, SSHRC Insight Development Grant, ``Reading with the Ears:
The Sounds of Violence in Contemporary Latin American Literature,''
\$46,897}

\cvitem{2023--24}{PI, Green College Interdisciplinary Thematic Series Fund,
\$10,000}

\cvitem{2022--24}{PI, ``Sound Studies and the Humanities Research Cluster,''
Public Humanities Hub Research Cluster, University of British Columbia,
\$16,000}

\cvitem{2023}{PI, Latin American Landscapes Visiting Speaker Grant, \$2000}

\cvitem{2023}{PI, ``Maize in Mayan AgriCulture: Workshop and Dialogue.''
Advancing Community Engaged Learning Fund, University of British
Columbia, \$1000}

\cvitem{2023}{Dean of Arts Graduate Mentorship Award, Faculty of Arts, University
of British Columbia}

\cvitem{2023}{PI, Faculty of Arts Undergraduate Researcher Award (AURA),
University of British Columbia, \$3000}

\cvitem{2022}{PI, Dorothy Dallas Funds, \emph{The Sounds of Latin American
Literature: Conference \& Workshop}, University of British Columbia,
\$2500}

\cvitem{2022}{SSHRC Exchange Workshop Grant, \emph{The Sounds of Latin American
Literature: Conference \& Workshop}, University of British Columbia,
\$2000}

\cvitem{2022}{``Latinocanadá: Writing from the Other Americas,'' SSHRC Explore
Grant, \$6000}

\cvitem{2021--22}{SSHRC Exchange Arts International Conference Travel Grant
(Washington DC), \$2000}

\cvitem{2021}{Work Learn International Undergraduate Researcher Award, University
of British Columbia, \$8000}

\cvitem{2021}{Faculty of Arts Undergraduate Researcher Award (AURA), University
of British Columbia, \$3000}

\cvitem{2021}{Canada Research Continuity Emergency Fund (CRCEF), \$2000}

\cvitem{2020--21}{SSHRC Exchange Arts International Conference Travel Grant
(Guadalajara, Mexico), \$2000}

\cvitem{2019--21}{Hampton New Faculty Research Grant, University of British
Columbia, \$10,000}

\cvitem{2020}{SSHRC Explore Grant for Research Assistant Support, University of
British Columbia, \$3000}

\cvitem{2020}{Digital Humanities Summer Institute Scholarship, University of
Victoria, Canada, \$1200 (declined)}

\cvitem{2019}{Latina/o Studies Section Travel Award, LASA (Boston, MA), \$500}

\cvitem{2018--19}{College of Arts and Sciences Dissertation Completion 
Fellowship, Indiana University, \$25,000}

\cvitem{2019}{Archival Research Grant: ``Roque Dalton: Declassified.'' Archivo
General de la Nación, Mexico City, Mexico. Center for Research on Race
and Ethnicity in Society, Indiana University, \$750}

\cvitem{2018}{ Timothy J. Rogers Summer Dissertation Fellowship, Department of
Spanish and Portuguese, Indiana University, \$5000}

\cvitem{2017}{ Alondra Nelson invited talk, Institute for Advanced Study Branigin
Lecture Fund, \$5000}

\cvitem{2017}{ Timothy J. Rogers Summer Dissertation Fellowship, Department of
Spanish and Portuguese, Indiana University, \$4000}

\cvitem{2016}{ Undergraduate Research Mentorship Program, IU Office of Engaged
Learning, \$9000}

\cvitem{2016}{ Nominee, Graduate of the Last Decade (GOLD) Award. Washburn
University, Topeka, KS}

\cvitem{2014--15}{ Academic Year Fellowship: Doctoral Student Academic
Achievement Award. Department of Spanish and Portuguese, Indiana University,
Tuition Waiver \& Teaching Release}

\cvitem{2015}{ Graduate Assistance in Areas of National Need (GAANN) Fellowship.
US Department of Education, \$15,000}

\cvitem{2014}{ Tinker Field Research Grant, archival research on Clarice
Lispector in Rio de Janeiro, Brazil. Center for Latin American and Caribbean
Studies, Indiana University, \$1500}

\cvitem{2014}{ Foreign Language Area Studies (FLAS) Fellowship, Portuguese
language study in Salvador de Bahia, Brazil. Center for Latin American
and Caribbean Studies, Indiana University, \$4425}

\cvitem{2014}{ Nominee, J.M. Hill Award for Outstanding Graduate Student Paper,
``Losing Our Marbles: The Technology of Literature in \emph{2666}.''
Department of Spanish and Portuguese, Indiana University}

\cvitem{2013--14}{ Teaching Fellowship. Latino Studies Program, Indiana
University, \$14,500}

\cvitem{2013}{ Best Graduate Student Essay, ``\,`La guagua aérea': Puerto Rican
Placeholder.'' Latino Studies Program, Indiana University, \$500}

\cvitem{2013}{ Graduate Top-up Fellowship. Department of Spanish and Portuguese,
Indiana University, \$3500}

\section{Select presentations}

\cvitem{}{\textbf{Invited Talks and Lectures}}

\cvitem{2025}{``Muffled Protest: Silence as Activism.'' MexicanEasts, Brigham
Young University, UT, September}

\cvitem{}{``Sound Unseen: Auscultating Gendered Violence in Mexican Fiction.''
Department of Spanish and Portuguese Lecture Series, University of
Toronto, March}

\cvitem{2024}{ ``Echoes of a Migrant Crisis: Narrative Attunement and Sound as
Ethics in 21st-Century Border Novels.'' Reed College, Portland, OR,
April}

\cvitem{}{``Listening in the Anthropocene: Migrant Sonority and More-than-Human
Perspectivism in Emiliano Monge's \emph{Las tierras arrasadas}.''
\emph{El coloquio interdisciplinario}, Department of Languages and
Cultures, Western University Ontario, March}

\cvitem{}{``Sound Unseen: Auscultating Gendered Violence in Mexican Fiction.''
\emph{Green College Lecture Series}, University of British Columbia,
February}

\cvitem{2023} {``Necropolitical Romeo and Juliet: Migrant Sonority against
Dehumanization in Emiliano Monge's \emph{Las tierras arrasadas}.''
University of Cincinnati, \emph{Taft Public Lecture Series}, March}

\cvitem{}{Roundtable discussion. ``Weeping Women: La Llorona's Presence in Modern
Latinx and Chicanx Lore.'' Association of Writers \& Writing Programs
Conference, March}

\cvitem{2022}{ ``In the Key of Crisis: Intermediality and Economic Downturn in
Laury Leite's \emph{En la soledad de un cielo muerto}.'' Iberian and
Latin American Cultures Lecture Series, Stanford University, May}

\cvitem{2021} {``Conspicuous Impunity: Ayotzinapa Seven Years Later.'' Hispanic
Heritage Month: Urgent Issues in Latin America Roundtable, Latin
American Studies Program, UBC, October (virtual)}

\cvitem{}{``In the Key of Crisis: Intermediality and the Romanticization of
Labour in Mexico City.'' Mexican Studies Research Collective, September
(virtual)}

\cvitem{}{``Academia Without Borders.'' Crisis Global, Desigualdades y
Centralidad de la Vida, XXXIX Latin American Studies Association Conference,
LASA Student Section, May (virtual)}

\cvitem{}{``Claudia Hernández's Visceral Realism and Central American Post-War
Fiction,'' LIT124B: The Contemporary Latin American Short Story,
Department of Literature, UC Santa Cruz, March}

\cvitem{2020}{ ``La esclavitud, la maquila y la alegoría fracasada: Un
renacimiento de Harlem en la Ciudad Juárez.'' UC Santa Cruz Spanish
Studies Colloquium, November (virtual)}

\cvitem{}{``Solidarity, Foodways, and Fierce Care on the Migrant Trail.'' Global
Resource Systems Program, University of British Columbia, November
(virtual)}

\cvitem{}{``In the Wake: Central American Literature after Oscar Romero.'' UBC
Latin American Studies Program (cancelled, COVID), March}

\cvitem{}{`` `La violencia es macro:' Natalie Beristáin's \emph{Los adioses}.''
UBC Latin American Studies Program, January}

\cvitem{2019}{ ``Trauma, Affect, and Testimonio in Horacio Castellanos Moya's
\emph{Insensatez}.'' Department of Spanish \& Portuguese, Indiana
University, April}

\vspace*{6pt}
\cvitem{}{\textbf{Conference and Workshop Presentations} (including participation
by invitation)}

\cvitem{2025}{ ``Escuchar contra la extinción: La auralidad como acción en la
poesía mexicana contemporánea.'' Congreso Internacional de
Americanistas, Novi Sad, Serbia, July}

\cvitem{}{``Uncaging the Literary Songbird: Aurality against Extraction in
Contemporary Mexican Poetry.'' XXIX Annual Juan Bruce-Novoa Mexican
Studies Conference, UC Irvine, May}

\cvitem{}{``Oímos cantar al pirirí: Attuning to the More-than-Human in Mexican
Literature.'' \emph{Sound and Literature Now} (sponsored panel, Sound
Forum). Modern Language Association Annual Conference, New Orleans,
January}

\cvitem{2024}{ ``Silent Toxicity: Extractivism and Machismo in Jorge Comensal's
\emph{Mutaciones}.'' XLII Latin American Studies Association Conference,
Bogotá, June}

\cvitem{}{``La comunidad única: Listening as Community and Care in Guadalupe
Nettel.'' XXVIII Annual Juan Bruce-Novoa Mexican Studies Conference, UC
Irvine, April}

\cvitem{2023}{ ``Sound Unseen: Auscultating Gendered Violence in Guadalupe
Nettel.'' XLI Latin American Studies Association Conference, Vancouver,
May}

\cvitem{}{``Precarious Orality: The Sounds of Forced Migration in Two Mexican
Novels.'' Sponsored Session, ``The Working Conditions of Sound
Studies,'' Presidential Theme. Modern Language Association Annual
Conference, San Francisco, January}

\cvitem{2022}{ ``From \emph{deshumanización} to \emph{desapropiación}: Central
American Migration in the Mexican Novel.'' Asociación Canadiense de
Hispanistas, June (virtual)}

\cvitem{}{``The Sounds and Spatiality of Crisis in Valeria Luiselli's \emph{Lost
Children Archive}.'' XL Latin American Studies Association Conference,
San Francisco, May (virtual)}

\cvitem{}{``In the Key of Crisis: Intermediality and Economic Migration in Laury
Leite's \emph{En la soledad de un cielo muerto}.'' XXVI Annual Juan
Bruce-Novoa Mexican Studies Conference, UC Irvine, April}

\cvitem{}{``Marking Time: The Sounds of Neoliberal Temporality at the Mexico-U.S.
Border.'' Modern Language Association Annual Conference, Washington DC,
January (cancelled, COVID)}

\cvitem{2021}{ ``Left Decadence and Intertextuality in Horacio Castellanos Moya's
(Post-)War Fiction.'' \emph{Central American Literature as World
Literature}, Rice University, November (virtual)}

\cvitem{}{``A Sensual Palimpsest: History, Memory and the Sonic in Laury Leite's
\emph{En la soledad de un cielo muerto}.'' XXXIX Latin American Studies
Association, May (virtual)}

\cvitem{}{``\href{https://youtu.be/nDUspX0GRlo?t=1425}{Rewriting Shit}: Dirt and
Excrement in Latin American Indigenous Literature.'' \emph{Building
Worlds}, Green College Leading Scholars Series, UBC, January (virtual)}

\cvitem{2020}{ ``A Harlem Renaissance in Ciudad Juárez: Race, Gender, and a
Crisis of Intersectionality in Roberto Bolaño's `La parte de Fate.'\,'' Améfrica
Ladina, XXXVIII Latin American Studies Association, Guadalajara, México
(cancelled, COVID), May}

\cvitem{}{``From the Salvadoran Civil War to the Mexico-US War on Drugs:
Neoliberal Narratives of the Central American(-American) Postwar.''
Latino Studies Association Biannual Conference, Notre Dame University
(cancelled, COVID), July}

\cvitem{2019}{ ``Broken Bodies, Broken Nations: Roberto Bolaño on Neoliberal
Logic and (Un)Mediated Violence.'' Hispanic Studies Seminar, FHIS, University of
British Columbia, October}

\cvitem{}{``Dissensus and Alternative Left Politics in Postwar El Salvador: Jazz
as Metonymy in Horacio Castellanos Moya's \emph{La diáspora.}'' LV
Congreso de la Asociación Canadiense de Hispanistas, Vancouver, Canada,
June}

\cvitem{}{``Nuestra América through a Latinx Lens: Reading (with) Héctor Tobar.''
Roundtable with the author. XXXVII Latin American Studies Association
Conference, Boston, MA, May}

\cvitem{}{``Specters of Roque: The Pen and/as the Sword.'' Selected for early
acceptance. XXXVII Latin American Studies Association Conference,
Boston, MA, May}

\cvitem{}{``Violence as Lingua Franca: Reading Bolaño on Neoliberalism, the
Failed State, and Political Precarity.'' American Comparative Literature
Association Annual Meeting, Washington DC, March}

\cvitem{}{``Precursors of 1968: Borges, Bolaño, and the Latin American Left.''
\emph{Special Session}, Textual Transactions, Modern Language
Association Annual Conference, Chicago, January}

\cvitem{2018}{ ``Broken Bodies, Broken Nations: Roberto Bolaño's \emph{2666} and
Neoliberal Consumption.'' \emph{Special Session}, Midwest Modern
Language Association, Kansas City, MO, November}

\cvitem{}{``Geopoetics, Geopolitics, and Global Violence: (Un)Mapping Daniel
Alarcón's \emph{Lost City Radio}.'' Latino Studies Association Biannual
Conference, Washington, DC, July}

\cvitem{}{``Crossing Borders, Literary and Political: Horacio Castellanos Moya's
Post-National Literature.'' XXXVI Latin American Studies Association
Conference, Barcelona, Spain, May}

\cvitem{2017}{ ``Disappearing Nations, Disappearing Genres: The Decline of the
Modern State and Post-National Literature,'' Ohio Latin Americanist
Conference, The Ohio State University, October}

\cvitem{}{``The Body in Crisis: Reading Bolaño on Neoliberalism, the Failed
State, \& Political Precarity.'' Diálogos de saberes, LASA XXXV Conference,
Lima, Peru, April}

\cvitem{2016}{ ``Snake Bites and Bestiality: \emph{Baile con serpientes} and the Return of
Magical Realism.'' I $\varheartsuit$ POP, Interdisciplinary Conference of the
Department of Comparative Literature, Graduate Center, City University of New
York (CUNY), November}

\cvitem{}{``Violence and Geopolitical Universality in Daniel Alarcón's \emph{Lost
City Radio}.'' Crossing the Lines: 12\textsuperscript{th} Annual
``Samuel G. Armistead'' Colloquium, UC Davis, October}

\cvitem{}{``Crossing the Line: Abjection and Social Permeability in \emph{Los
ríos profundos}.'' CRRES Graduate Student Research Symposium, Indiana University
Bloomington, April}

\cvitem{}{``Violencia y alienación social: \emph{El arma en el hombre} como
Bildungsroman neoliberal.'' XXIV Congreso Internacional de Literatura
Centroamericana, Cáceres, Spain, April}

\cvitem{2015}{ ``Apocryphal Memory and Displacing the Chronological Past:
Reconfigurations of Time in Bolaño's \emph{Amuleto}.'' LASA:
Precariedades, exclusiones, emergencias, XXXIII Latin American Studies
Association Conference, San Juan, Puerto Rico, May}

\cvitem{}{``Los expulsados se exceden: La escatología y la permeabilidad social
en \emph{Los ríos profundos}.'' XX Charles F. Fraker Conference, University of
Michigan, Ann Arbor, March}

\cvitem{2014}{ ``Writing and Reading the Dead in Oscar Zeta Acosta: A Historical
Materialist Approach to Literary Corpses.'' Tierra Tinta X Conference,
University of Oklahoma, October}

\cvitem{2013}{ ``La pluma bifurcada: Materialismo histórico en \emph{El reino de
este mundo}.'' Mid-America Conference on Hispanic Literatures,
University of Missouri, November}

\cvitem{}{`` `La guagua aérea': The Puerto Rican Placeholder.'' Imagined Spaces:
Kaleidoscope Graduate Student Conference, University of
Wisconsin-Madison, March}

\cvitem{2012}{ ``Duœling Discourses: Luis Rafael Sánchez's \emph{Quíntuples}.''
10th Annual Hawaii International Conference on Arts and Humanities,
January}

\vspace*{6pt}
\cvitem{}{\textbf{Interviews}}

\cvitem{2021}{``\href{http://thelasource.com/en/2021/01/11/new-years-goal-changing-our-feelings-about-poop/}{Next
Year's Goal}: Changing our Feelings about Poop,'' The/La Source, Vancouver,
January}

\cvitem{2019}{ ``\href{https://fhis.ubc.ca/news/research-spotlight-post-nationalism-in-mexican-and-central-american-narrative-fiction/}{Research
Spotlight}: Post-Nationalism in Mexican and Central American Narrative
Fiction,'' FHIS Newsletter, Vancouver, November}

\cvitem{2014}{``\href{https://wfhb.org/news/httpwfhb-orgwp-contentuploadshola-20141212-mp3/}{Ayotzinapa
43},'' HOLA Bloomington (WFHB), Bloomington, IN, November}

\vspace*{6pt}
\cvitem{}{\textbf{Conferences, symposia, and sessions organized}}

2025 Organizer, ``Resonant Returns: Echoes of the Mexican Canon in
Contemporary Aural Culture.'' XXIX Annual Juan Bruce-Novoa Mexican
Studies Conference, UC Irvine, May

2024 Co-Organizer (with Isabella Vergara, Northwestern University),
``Espacios sonoros: El silencio como expresión en la literatura y la
cultura latinoamericanas contemporáneas'' (joint session), XLII Latin
American Studies Association Conference, June

2023 Organizer, invited lecture by Prof. Carolina Sá Carvalho
(University of Toronto), ``How to See a Scar,'' FHIS Research Seminar,
November

Organizer, ``Sensing (In)Justice in Feminist Literature: Listening
in/against the Lettered City,'' XLI Latin American Studies Association
Conference, May

Co-Organizer (with Dylan Robinson and Julen Etxabe), invited lecture by
Prof. Trevor Reed (Arizona State University), ``Restorative Justice for
Indigenous Voices,'' March

2022 Organizer, invited lecture by Prof. Tom McEnaney (U. of California,
Berkeley), ``Stop Listening: Surveillance, Sound, Sexuality, and Race in
Manuel Puig's \emph{Kiss of the Spider Woman},'' November

Co-Organizer (with Prof. Amanda Smith, UC Santa Cruz),
\emph{\href{https://blogs.ubc.ca/sonidolit/}{Listening with the Eyes}:
The Sounds of Latin American Literature Conference \& Workshop},
University of British Columbia, October

Joint Session, Co-Organizer (with Prof. Ali Kulez, Boston College),
``Reframing Crisis in Contemporary Latin American Literature.''
\emph{Polarización socioambiental y rivalidad entre grandes potencias},
XL Latin American Studies Association Conference, May

2021 Co-Organizer (with Prof. Amanda Smith, UC Santa Cruz), ``Sensible
Disruptions: Soundscapes in Contemporary Latin American Literature.''
\emph{Crisis global, desigualdades y centralidad de la vida}, XXXIX
Latin American Studies Association Conference, May

Session Co-Organizer, ``Waste Not: Rethinking Poop through Bugs, Books
and Power.'' \emph{Building Worlds}, Green College Leading Scholars
Series, UBC, January

Session Organizer, ``Listening in Latin America: Narrative Soundscapes
\& Literary Aurality.'' Modern Language Association Annual Convention:
\emph{Persistence}, Toronto, January

2020 Session Organizer, ``Central American Civil Conflict at a Distance:
Narrative Reflections from the Diaspora.'' Latina/o Studies Association
4th Biennial Conference, July

Session Organizer, Prof. Patrick Dove (Indiana University),
``Biopolitics, COVID, and Latin American Literature.'' Virtual Koerner's
Reading Group, July

Session Organizer, ``Géneros: Gender and Genre in the Literatures of
Améfrica Ladina.'' XVIII Latin American Studies Association,
Guadalajara, México, May

2019 Session Organizer, ``Nuestra América through a Latinx Lens: Reading
(with) Héctor Tobar.'' Roundtable with the author. Nuestra América,
XXXVII Latin American Studies Association Conference, Boston, MA, May

Session Organizer, ``Residues of the Archive: On Belonging, Inequality,
and Crisis from the IsthmUS.'' Selected for early acceptance. Nuestra
América, XXXVII Latin American Studies Association Conference, Boston,
MA, May

Co-Organizer (with Prof. Matt Johnson, New Mexico Tech), Presidential
Session, ``Borgesian Transactions: Literary Debts, Literary
Inheritances,'' MLA Annual Convention, Chicago, January

2018 Special Session Organizer, ``Consumption of the Other, Consumption
of the Self: Aesthetic Mutilation as Neoliberal Critique.'' Midwest
Modern Language Association Conference, Kansas City, MO, November

Session Organizer, ``Political and Literary Geographies in Flux: Culture
and Media in the Era of Post-National Globalization.'' Latin American
Studies in a Globalized World, Latin American Studies Association XXXVI
Conference, Barcelona, Spain, May

Symposium Organizer, \emph{Contested Spaces, Contested Identities},
Center for Research on Race and Ethnicity in Society, Indiana
University, Bloomington, April

2017 Session Organizer, ``Impactos del capitalismo globalizado:
Fenómenos literarios glocales desde una perspectiva intercontinental,''
Ohio Latin Americanist Conference, Ohio State University, October

Session Organizer, ``Culture on Crisis/Culture in Crisis: Art and the
Neoliberal Market.'' Diálogos de saberes, XXXV Latin American Studies
Association Conference, Lima, Peru, April

2014 Organizer, Diálogos 11: Graduate Student Research Conference,
Department of Spanish and Portuguese, Indiana University, February

Symposium Co-Organizer (with Gaëlle Le Calvez), \emph{The Politics of
Violence in Mexico: A Public Forum for Information, Solidarity, \&
Action} (Global Forum, \emph{México y la Herida del Mundo}), IUB,
November

\section{Mentorship and supervision}

\textbf{PhD Supervisor} (\emph{year indicates actual or anticipated
completion})

2028 Ester Aguirre Alfaro, French, Hispanic and Italian Studies

2027 Fabiola del Rincón Fernández, French, Hispanic and Italian Studies

2026 Sarah Revilla-Sánchez, French, Hispanic and Italian Studies

2026 Pamela Zamora Quesada (UBC Affiliated Fellow), French, Hispanic and
Italian Studies

2026 Kathryn Houston (SSHRC Fellow), French, Hispanic and Italian
Studies

\vspace*{6pt}
\textbf{PhD Committee Member}

2026 Lorenia Salgado-Leos, French, Hispanic and Italian Studies

2026 Santiago Farias Calderón, Department of Theatre and Film

2023 Ricardo García Martínez, French, Hispanic and Italian Studies
(dissertation defence, August 2023)

\vspace*{6pt}
\textbf{MA Thesis Supervisor}

2025 Keira Smith-Tague (SSHRC Fellow), French, Hispanic and Italian
Studies

2025 Mirella Reichenbach Livoti (UBC Affiliated Fellow), French,
Hispanic and Italian Studies

2023 Samuel Aguayo Mejía, French, Hispanic and Italian Studies

2022 María Fernanda Guzmán Rodríguez (UBC Affiliated Fellow), French,
Hispanic and Italian Studies

\vspace*{6pt}
\textbf{MA Thesis Committee}

2022 Johann Pitter, French, Hispanic and Italian Studies (thesis
defence, April 2022)

2022 Nancy Ross, French, Hispanic and Italian Studies (thesis defence,
August 2022)

\vspace*{6pt}
\textbf{Funded Research Assistants}

2025- Ester Aguirre Alfaro (PhD, FHIS), ``Sounds of the Capitalocene:
Extraction and Aurality in Mexican Literature,'' SSHRC Insight Grant

2025- Pamela Zamora Quesada (PhD, FHIS), ``Sounds of the Capitalocene:
Extraction and Aurality in Mexican Literature,'' SSHRC Insight Grant

2024 Pamela Zamora Quesada (PhD, FHIS), ``Public Humanities Engagement:
Podcast and Community Discussion Guides,'' Public Humanities Hub Cluster
Grant

2022-24 Sarah Revilla Sanchez (PhD, FHIS), ``Reading with the Ears: The
Sounds of Violence in Contemporary Latin American Literature,'' SSHRC
Insight Development Grant

2022-24 Pamela Zamora Quesada (PhD, FHIS), ``Reading with the Ears: The
Sounds of Violence in Contemporary Latin American Literature,'' SSHRC
Insight Development Grant

2023 Fabiola del Rincón Fernández (PhD, FHIS), ``More-than-Human
Sonority in Emiliano Monge''

2023 Keira Smith-Tagle (MA, FHIS), ``Latinocanadá: Indigeneity and
Literature of the Latin American Diaspora in Canada,'' Faculty of Arts
Undergraduate Researcher Award (AURA)

2021-22 Pamela Zamora Quesada (PhD, FHIS), ``Noisy Narratives Digital
Platform'' \& ``Ontologies of Namelessness,'' Hampton Fund New Faculty
Research Grant

2022 Kathryn Houston (PhD, FHIS), ``Latinocanadá: Writing from the Other
Americas,'' SSHRC Explore Grant for Research Assistant Support

2022 Daisy Sessions (BA, History, Latin American Studies),
``Latinocanadá: Writing from the Other Americas,'' SSHRC Explore Grant
for Research Assistant Support

2021 Daniela García (BA, International Relations), ``Maíces y Milpa'',
Work Learn International Undergraduate Researcher Award

2021 Grey Figueroa Mercado (BA, Spanish), ``Maíces y Milpa'', Faculty of
Arts Undergraduate Researcher Award (AURA)

2020 Pamela Zamora Quesada (PhD, FHIS), ``Agrarian Reform and the Novel
of the Mexican Revolution.'' Hampton Fund New Faculty Research Grant

2020 Lorenia Salgado-Leos (PhD, FHIS), ``Interrupted Intimacies:
Neoliberal Erotics in Central American Literature.'' SSHRC Explore Grant
for Research Assistant Support

2020 Ricardo García Martínez (PhD, FHIS), ``Latin American Literary
Soundscapes.'' Hampton Fund New Faculty Research Grant

\section{Teaching experience}

\cvitem{}{University of British Columbia Vancouver Assistant Professor,
\emph{2019--present}}
\cvitem{}{Department of French, Hispanic \& Italian Studies}

SPAN 590. Ecocriticism and the Anthropocene in Mexican and Central
American Literature (Su24)

SPAN/FREN 592. Proseminar II: Literary and Cultural Theory (Sp24)

SPAN/FREN 591. Proseminar I: Research Skills and Scholarly Practices
(Fa23)

SPAN 590. Narrativa Sonora: Sound and Literature in Contemporary Latin
American Fiction (Fa22)

SPAN 550. Latin(x) American Speculative Fictions in the Anthropocene
(Sp22)

SPAN 590. Sonidos textuales y escritos sonoros: Central American
Literary Soundscapes (Su21)

SPAN 550. Porn Lit: Critical Approaches to Erotic Literature in Latin
America and Spain (Sp21)

SPAN 490. Latin American Detective Fiction and the Nation State (Fa19)

SPAN 365. Modern Magics: Spanish-American Literature and Culture since
1820s (Sp20-Sp24)\\
TA (grad), Spring 2021: Pamela Zamora Quesada

SPAN 312. Hopscotch: Twentieth \& Twenty-first Century Latin American
Literature (Fa21)\\
TA (undergrad): Daniela García

SPAN 202. Advanced Beginner's Spanish II (Sp22, Sp24)

\cvitem{}{Latin American Studies Program}


LAST 315. Food and Literature in Mexico and Central America (Fa23)

\href{https://blogs.ubc.ca/course0935bb2c1345dfa4e91d0701421d97f5c03a0045/}{LAST
303}. Latin America Indigenous Foodways (Fa20, Sp23): Open access podcast
\href{https://blogs.ubc.ca/course0935bb2c1345dfa4e91d0701421d97f5c03a0045/lectures/}{lectures}

\href{https://blogs.ubc.ca/indigenouslabour/}{LAST 303}. Indigenous
Peoples, Globalization, and Labour in Latin America (Sp20)

LAST 100. Introduction to Latin American Studies (Fa21, Fa22)

\vspace*{6pt}
\cvitem{}{\textbf{Indiana University, Bloomington, IN} Graduate Teaching Assistant,
\emph{2012--2015}\vspace{-3pt}}
\cvitem{}{Department of Spanish \& Portuguese; Latino Studies Program}


HISP 280. Advanced Spanish Grammar \& Composition in Literary Context
(Fa15)

HISP 250. Basic Language IV (Sp13, Su13)

Hispanic Literatures I, II, III, Study Abroad courses in León, Spain
(Su13)

HISP 105. Accelerated Basic Language I (Fa12)

LATS 103. Introduction to Latino Cultures \& Literatures (Fa14, Sp15,
Fa15)

\vspace*{6pt}
\cvitem{}{\textbf{Clemson University, Clemson, SC} Lecturer, S202 Couse Coordinator,
\emph{2010--2012}\vspace{-3pt}}
\cvitem{}{Department of Modern Languages}
SPAN 102. Accelerated Basic Language I (4 sections, Fa10, Sp11)

SPAN 201. Basic Language III (4 sections, Fa11, Sp12)

SPAN 202. Basic Language IV (10 sections, Fa10-Sp12)

SPAN 398. Ecuadorian Literature and History (Su12)

International Studies 210. Ecuadorian Culture in Context (Su12)

Advanced Spanish Conversation, Summer Language Immersion Program (Su11)

\vspace*{6pt}
\cvitem{}{\textbf{University of Santiago de Compostela, Galicia, Spain} Assistant
Professor, \emph{2009--2010}\vspace{-3pt}}
\cvitem{}{Department of English Philology}

English Philology I. Pre-intermediate Language, Literature and Culture,
(Fa09, Sp10)

English Philology II. Intermediate Language, Literature and Culture
(Fa09, Sp10)

English Philology IV. Advanced Literature and Translation (4 sections,
Fa09, Sp10)

\vspace*{6pt}
\cvitem{}{\textbf{University of Kansas, Lawrence, KS} Graduate Teaching Assistant,
\emph{2007--2009}\vspace{-3pt}}
\cvitem{}{Department of Spanish and Portuguese}

SPAN 111. Accelerated Basic Spanish (4 sections: Fa07, Sp08, Fa08, Sp09)

SPAN 270. Literature, Culture and History of Spain, Barcelona, Spain
study abroad (Su08)

\section{Study abroad and immersion program coordination}

Indiana University, Honors Program in Foreign Languages\\
Logistics Coordinator, Instructor of Hispanic Literatures León, Spain,
Summer 2013

Clemson University, Department of Modern Languages\\
Study Abroad Director, Instructor of S398 and IS210 Quito, Ecuador,
Summer 2012\\
Director, Clemson Language Immersion Program (CLIP) Clemson, SC, Summer
2011

University of Kansas, Department of Spanish and Portuguese\\
Teaching Assistant, SPAN 270 Barcelona, Spain, Summer 2009

\section{Departmental and university service}

\bolditem{Service to the University of British Columbia}

2022-25 Steering Committee, Latin American Studies Program

2024 Membership Committee, Green College

2023-24 Organizer, Co-Convenor,
\emph{\href{https://sound.arts.ubc.ca/events/}{Sound Silence Power}}
Thematic Series, Sound and the Humanities Research Cluster. Invited
Speakers: Christina Sharpe (October 2023), Luis Cárcamo-Huechante
(November 2023), Casey Mecija (January 2024), Jonathan Sterne (April
2024)

2022-24 Spanish Curriculum Sub-Committee, Hispanic Studies Section,
Department of French, Hispanic \& Italian Studies

2022-23 Research Promotion Coordinator, Department of French, Hispanic
\& Italian Studies

2022-23 Chair, Faculty Awards Committee, Department of French, Hispanic
\& Italian Studies

2023 Member, FHIS Head Selection Committee, Faculty of Arts

2023 Discussion Facilitator \& Presenter, ``Latin American Indigenous
Studies in Theory.'' Department of French, Hispanic \& Italian Studies
Pro-Seminar

2022-23 Member, Portuguese Lecturer Search Committee, Department of
French, Hispanic \& Italian Studies

2022 Invited Presenter ``The Academic Job Market.'' Graduate Studies
Committee Professionalization Series, Department of French, Hispanic \&
Italian Studies

2020-22 Faculty Representative, UBC Imagine Day, Latin American Studies
Program

2019-22 Member, Graduate Studies Committee, Department of French,
Hispanic \& Italian Studies

2019-22 Member, Faculty Awards Committee, Department of French, Hispanic
\& Italian Studies

2021-22 Marking and Grading Guidelines Committee, Department of French,
Hispanic \& Italian Studies

2021 Invited Presenter, ``Encouraging Student Engagement and
Collaboration through UBC Blogs.'' Faculty of Arts, Instructional
Support and Information Technology

2021 Master of Ceremonies, Graduation Celebration, Department of French,
Hispanic \& Italian Studies

2021 Virtual Graduation Celebration Committee, Department of French,
Hispanic \& Italian Studies

2020-21 Steering Committee, ``Virtual Koerner's Working Group''

2019-21 Textbook Adoption Committee (Spanish Basic Language), French,
Hispanic \& Italian Studies

2019-21 Minute-Taker, Hispanic Studies Section, Department of French,
Hispanic \& Italian Studies

2020 Invited Presenter and CV Workshop Facilitator, ``The Academic Job
Market.'' Graduate Studies Committee Professionalization Series,
Department of French, Hispanic \& Italian Studies

\vspace*{6pt}
\bolditem{Service to Indiana University Bloomington}

2013-18 Graduate Student Mentor, Department of Spanish and Portuguese

2016-18 Mentor, CRRES Undergraduate Research Program in the Social
Sciences and Humanities: Mentorship of underrepresented minority
students in research practices, professionalization

2016 Discussion Facilitator, ``Immigration, Identity, \& Exclusion,''
Provost's \emph{Hot Topics} Series

2016 Session Presenter and Discussant, ``Thinking with Derrida: A
Symposium,'' Center for Theoretical Inquiry in the Humanities

2014-16 Essay Judge, ``César E. Chávez Undergraduate Essay
Competition,'' Latino Studies Program

2013-16 Mentor, ``César E. Chávez Undergraduate Research Symposium,''
Latino Studies Program

2014 Film Presenter, \emph{Gloria} (2014, Chile). Department of Spanish
and Portuguese Film Cycle

2014 Film Presenter, \emph{La lucha de Ana} (2012, Dominican Repubic).
Latino Studies Film Festival

2013-14 Co-Chair, Graduate Student Advisory Committee, Department of
Spanish and Portuguese

2013 Member, S100 Spanish Materials Creation Committee, Department of
Spanish and Portuguese

\section{Professional and community service}

Executive Committee, Sound Forum, Modern Language Association (MLA),
2024-2029

Invited Presenter, ``Aural Borders and Ambiguous Sound: \emph{Echoes
from the Borderlands},'' (workshop on sound installations and aural
culture), Princeton University, March 2025

Invited Presenter, ``Latin American Sound Studies \& Research
Collaboration,'' Northwestern University, 2024

Member (elected), Standing Committee / ``Consejo Directivo,'' Latin
American Studies Association (LASA) Mexico Section, 2021-2023

Chair, ``Best Book in the Humanities Award Selection Committee,'' LASA
Mexico Section, 2022-2023

Discussant, ``Political Violence and Trauma in Hispanic Literatures,''
\emph{Impending Catastrophes Through the Ages: Literature and the Arts
in the Context of Doom}, 10th Biennial International Graduate Student
Conference, FHIS, October 2023

Discussant, ``Embodied Sensing in 20th and 21st-Century Mexican Literary
and Cultural Production,'' XLI Latin American Studies Association
Conference, May 2023

Discussant, ``Sensing (In)Justice in Feminist Literature: Listening
in/against the Lettered City,'' XLI Latin American Studies Association
Conference, May 2023

Discussant, FHIS Graduate Student Symposium, ``Race and Melancholia in
Cervantes's Novelas'' April 2023

Editorial Board, \emph{Chiricú Journal: Latina/o Literatures, Arts, and
Cultures}, Indiana University Press, 2019-2022

Invited Presenter, ``El mercado laboral,'' Mexican Studies Research
Collective, August 2022

Organizer, Facilitator, ``Hispanic Studies Young Faculty Working
Group,'' 2020-present

Discussant, ``Reframing Crisis in Contemporary Latin American
Literature.'' \emph{Polarización socioambiental y rivalidad entre
grandes potencias}, XL Latin American Studies Association Conference,
May 2022

Discussant, FHIS Graduate Student Symposium, ``Oralidades y
textualidades,'' May 2022

Member, ``Dissertation Award Selection Committee,'' LASA Mexico Section,
2021-2022

Co-Organizer, Facilitator, ``Latin American Sound Studies Reading
Group,'' 2021-present

Discussant, ``Ancestral Persistence and Care,'' \emph{In-Between
Normalities. Care, Persistence, and the (Re)Imagination of Life}, 9th
Biennial International Graduate Student Conference, FHIS, October 2021

Invited Speaker and Workshop Facilitator, ``The Academic Job Market.''
Professionalization Series, Department of Spanish and Portuguese,
Indiana University Bloomington, November 2020

Discussant, FHIS Graduate Student Symposium, ``A Place for Thinking,''
April 2021

Discussant, ``Listening in Latin America: Narrative Soundscapes \&
Literary Aurality'' Modern Language Association Annual Convention:
\emph{Persistence}, Toronto, January 2021

Discussion Facilitator, \emph{Reading Group: Latin American Short
Stories}, Vancouver Latin American Cultural Centre, 2020-present

Invited Juror, ``Lista Arcadia de los libros del año: Autoras
hispanoamericanas'' (``Books of the Year, Arcadia: Women Novelists''),
Editorial Arcadia, October 2019

Discussant, ``Crossing Borders'' VIII Biennial International Graduate
Student Conference, Department of French, Hispanic \& Italian Studies,
University of British Columbia, October 2019

Discussant, \emph{History and its Shadows: Re-Thinking Historical
Narratives in Contemporary Latin American and Luso-Brazilian
Literature}. Graduate Student Research Conference: Diálogos 13, Indiana
University, February 2016

Discussant, \emph{Luso-Brazilian Literature \& Culture}. Graduate
Student Research Conference: Diálogos 10, Indiana University, February
2013

\section{Manuscript Referee}

\textbf{Articles \& Chapters}

\emph{Letras Hispanas}

\emph{Bloomsbury Academic}

\emph{Journal of Latin American Cultural Studies}

\emph{Revista de Estudios Hispánicos}

\emph{Revista de Literatura Mexicana Contemporánea}

\emph{Canadian Journal of Latin American and Caribbean Studies}

\emph{Chiricú Journal: Latina/o Literatures, Arts, and Cultures}

\emph{Latin} \emph{American Literary Review}

\emph{Arizona Journal of Hispanic Cultural Studies}

\section{Related experience and training}

2019-21 Green College Leading Scholars Program, University of British
Columbia

2019 Workshop Participant (with Horacio Castellanos Moya), ``Horacio
Castellanos Moya's~\emph{El sueño del retorno}.'' Department of Spanish
and Portuguese, Indiana University Bloomington

2018 Workshop Participant (with Carmen Boullosa), ``Carmen Boullosa y
los escritores mexicanos después de 1968.'' College Arts \& Humanities
Institute, Indiana University Bloomington

2016-18 Graduate Research Assistant and Undergraduate Research Program
Coordinator, Center for Research on Race and Ethnicity in Society
(CRRES), Indiana University, Bloomington.

2016 Editorial Associate, \emph{Chiricú Journal: Latina/o Literatures,
Arts, and Cultures}. Indiana University Press

2014 Diálogo Portuguese Language School, Salvador de Bahia, Brazil:
Study Abroad

2014 Translator, Chestnut Health Systems: ``Global Appraisal of
Individual Needs.'' English to Spanish

2013 Course Developer \& Spanish-language Editor, Option Six of GP
Strategies, Bloomington, IN

2012 Pedagogy Presenter, ``Name Game: First-Day Icebreaker.''
\emph{Share Fair: From Foreign to Familiar}

2012 Materials Creator, Cengage Publishing: Syllabus and Testing
Materials Creation

2012 Copy Editor, \emph{Italian Post-Neorealist Cinema} (published 2013)

2011 Non-degree-seeking graduate studies, Literary Theory (English
Department), Clemson University

2009-10 Translator (Spanish to English): Universidade de Santiago de
Compostela, Galicia, Spain

2006 La Universidad de Cantabria, Santander, Spain: Study Abroad

2004 Cambridge University, Cambridge, England: Study Abroad

\section{Academic affiliations}

Modern Language Association (MLA) 2011--present

Latin American Studies Association (LASA) 2012--present\\
Sections: Central America; Culture, Power and Politics; Latino Studies;
Mexico

American Comparative Literature Association (ACLA) 2016--present

Asociación Canadiense de Hispanistas (ACH) 2017-present

Green College Common Room, University of British Columbia 2019--present

Public Humanities Hub, University of British Columbia 2019--present

Mexican Studies Research Collective 2020-present

\section{Languages and proficiencies}

\textbf{Languages}

English: Native fluency Portuguese: Conversational fluency\\
Spanish: Near-native fluency French: Reading knowledge

\vspace*{6pt}
\textbf{Digital Humanities Proficiencies}

Programming (proficient): markdown, HTML, CSS\\
Expertise in social media, blogging, and podcast development\\
Software: Audacity, InDesign, Photoshop\\
Digital Publishing \& Communications: Github, WordPress

\end{document}